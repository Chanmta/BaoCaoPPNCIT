\small
\begin{textblock*}{210mm}(5mm,30mm)
\begin{tabular}{cc}
	HỌC VIỆN KỸ THUẬT QUÂN SỰ  & \textbf{CỘNG HÒA XÃ HỘI CHỦ NGHĨA VIỆT NAM} \\
	\textbf{KHOA CÔNG NGHỆ THÔNG TIN}  & Độc lập - Tự do - Hạnh phúc \\
	& \rule{130px}{1px}\\
	& \\
	\large\textbf{Phê chuẩn} & \large Độ mật: 8150 \\
	\large Ngày 07 tháng 12 năm 2019 & \large Số: 5120 \\
	\textbf{CHỦ NHIỆM KHOA} & \\
\end{tabular}
\end{textblock*}

\textbf{}\\\\\\\\\\\\\\\\\\\\\\\\\\

\begin{center}
	\textbf{\large NHIỆM VỤ ĐỒ ÁN TỐT NGHIỆP}\\
\end{center}

\textbf{}\\\\

\large{
	\noindent Họ và Tên: Lý Văn Chản, Nguyễn Ngọc Khánh, Bùi Đình Thủy\\
	Lớp: CNDL - Khóa: 15\\
	Ngành: Công nghệ thông tin\\
	Chuyên ngành: Công nghệ dữ liệu\\\\
	1. Tên đề tài: Nhận diện các phương tiện giao thông đi ngược chiều qua camera lắp đặt cố định tại một tuyến đường\\\\\\
	2. Các số liệu ban đầu: ...\\\\\\
	3. Nội dung bản thuyết minh: ...\\\\\\
	4. Số lượng, nội dung các bản vẽ (ghi rõ loại, kích thước và cách thực hiện các bản vẽ) và các sản phẩm cụ thể (nếu có): ...\\\\\\
	5. Cán bộ hướng dẫn:\\
	\indent Họ và Tên: Trần Cao Trưởng\\
	\indent Cấp bậc: Tiến Sĩ\\
	\indent Học hàm, Học vị: GV, T.S\\
	\indent Đơn vị: Bộ môn Khoa học máy tính\\\\\\
}

\hskip-2.0cm
\large
\begin{tabular}{cc}
	Ngày giao: 01/12/2019 & Ngày hoàn thành: 07/12/2019\\
	&\textit{Hà Nội, ngày 07/12/2019}\\
	& \\
	\textbf{Chủ nhiệm bộ môn} & \textbf{Cán bộ hướng dẫn}\\
	& (Ký, ghi rõ họ tên, học hàm, học vị)\\
	& \\
	& \\
	& \\
	& \\
	& \\
	& \textbf{Học viên thực hiện}\\
	& Đã hoàn thành và nộp đồ án ngày 07/12/2019\\
	& (Ký và ghi rõ họ tên)
\end{tabular}
